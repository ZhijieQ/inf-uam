monta un servidor T\-C\-P con conexion S\-S\-L\hypertarget{cerrar_canal__s_s_l_SYNOPSIS}{}\section{S\-Y\-N\-O\-P\-S\-I\-S}\label{cerrar_canal__s_s_l_SYNOPSIS}
{\bfseries \#include} {\bfseries \char`\"{}conexion.\-h\char`\"{}} 

{\bfseries S\-S\-L$\ast$} {\bfseries aceptar\-\_\-canal\-\_\-seguro\-\_\-\-S\-S\-L} {\bfseries }({\bfseries S\-S\-L\-\_\-\-C\-T\-X$\ast$} ctx, {\bfseries cint} sockfd, {\bfseries int} puerto, {\bfseries int} tam, {\bfseries struct} sockaddr\-\_\-in ip4addr{\bfseries })\hypertarget{cerrar_canal__s_s_l_descripcion}{}\section{D\-E\-S\-C\-R\-I\-P\-C\-IÓ\-N}\label{cerrar_canal__s_s_l_descripcion}
Dado un contexto S\-S\-L y un descriptor de socket esta función se encargará de bloquear la aplicación, que se quedará esperando hasta recibir un handshake por parte del cliente.\hypertarget{cerrar_canal__s_s_l_retorno}{}\section{R\-E\-T\-O\-R\-N\-O}\label{cerrar_canal__s_s_l_retorno}
Devuelve N\-U\-L\-L en caso de error o la estructura S\-S\-L$\ast$ inicializada en caso de exito.\hypertarget{cerrar_canal__s_s_l_seealso}{}\section{V\-E\-R T\-A\-M\-B\-IÉ\-N}\label{cerrar_canal__s_s_l_seealso}
{\bfseries inicializar\-\_\-nivel\-\_\-\-S\-S\-L(3)}, {\bfseries fijar\-\_\-contexto\-\_\-\-S\-S\-L(3)}, (3), {\bfseries evaluar\-\_\-post\-\_\-connectar\-\_\-\-S\-S\-L(3)}, {\bfseries enviar\-\_\-datos\-\_\-\-S\-S\-L(3)}, {\bfseries recibir\-\_\-datos\-\_\-\-S\-S\-L(3)}, {\bfseries cerrar\-\_\-canal\-\_\-\-S\-S\-L(3)} {\bfseries }  authors A\-U\-T\-O\-R Mario Valdemaro Garcia Roque (\href{mailto:mariov.garcia@estudiante.uam.es}{\tt mariov.\-garcia@estudiante.\-uam.\-es}) Roberto Garcia Teodoro (\href{mailto:roberto.garciat@estudiante.uam.es}{\tt roberto.\-garciat@estudiante.\-uam.\-es}) 