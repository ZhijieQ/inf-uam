envia un mensaje\hypertarget{cerrar_canal__s_s_l_SYNOPSIS}{}\section{S\-Y\-N\-O\-P\-S\-I\-S}\label{cerrar_canal__s_s_l_SYNOPSIS}
{\bfseries \#include} {\bfseries \char`\"{}conexion.\-h\char`\"{}} 

{\bfseries int} {\bfseries enviar\-\_\-datos\-\_\-\-S\-S\-L} {\bfseries }({\bfseries S\-S\-L$\ast$} ssl, {\bfseries const} coid $\ast$ buf{\bfseries })\hypertarget{cerrar_canal__s_s_l_descripcion}{}\section{D\-E\-S\-C\-R\-I\-P\-C\-IÓ\-N}\label{cerrar_canal__s_s_l_descripcion}
Esta función será el equivalente a la función de envío de mensajes que se realizó en la práctica 1, pero será utilizada para enviar datos a través del canal seguro. Es importante que sea genérica y pueda ser utilizada independientemente de los datos que se vayan a enviar.\hypertarget{cerrar_canal__s_s_l_retorno}{}\section{R\-E\-T\-O\-R\-N\-O}\label{cerrar_canal__s_s_l_retorno}
Devuelve 0 en caso de error o el tamaño del mensaje enviado en caso exito.\hypertarget{cerrar_canal__s_s_l_seealso}{}\section{V\-E\-R T\-A\-M\-B\-IÉ\-N}\label{cerrar_canal__s_s_l_seealso}
{\bfseries fijar\-\_\-contexto\-\_\-\-S\-S\-L(3)}, {\bfseries inicializar\-\_\-nivel\-\_\-\-S\-S\-L(3)}, {\bfseries conectar\-\_\-canal\-\_\-seguro\-\_\-\-S\-S\-L(3)}, {\bfseries aceptar\-\_\-canal\-\_\-seguro\-\_\-\-S\-S\-L(3)}, {\bfseries evaluar\-\_\-post\-\_\-connectar\-\_\-\-S\-S\-L(3)}, {\bfseries enviar\-\_\-datos\-\_\-\-S\-S\-L(3)}, {\bfseries recibir\-\_\-datos\-\_\-\-S\-S\-L(3)}, {\bfseries cerrar\-\_\-canal\-\_\-\-S\-S\-L(3)} {\bfseries }  authors A\-U\-T\-O\-R Mario Valdemaro Garcia Roque (\href{mailto:mariov.garcia@estudiante.uam.es}{\tt mariov.\-garcia@estudiante.\-uam.\-es}) Roberto Garcia Teodoro (\href{mailto:roberto.garciat@estudiante.uam.es}{\tt roberto.\-garciat@estudiante.\-uam.\-es}) 